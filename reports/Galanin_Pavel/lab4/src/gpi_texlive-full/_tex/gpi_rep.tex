\ESKDstyle{empty}

% = = = = = = = = = = = = = = = =

Цель работы:
ознакомиться с возможностями, предлагаемыми Qt для поддержки сетевого взаимодействия программ

% % = = = = = = = = = = = = = = = =

\section{Сервер}

Сервер возвращает JSON с версиями и dll файлы.

\begin{itemize}
    \item \url{http://localhost:3002/versions} - получаем JSON
    \item \url{http://localhost:3002/gpi_helper.dll} - загрузка dll
    \item \url{http://localhost:3002/gpi_helper_class.dll} - загрузка dll
    \item \url{http://localhost:3002/gpi_about.dll} - загрузка dll
\end{itemize}

\lstinputlisting[]{../gpi_server/package.json}

\lstinputlisting[]{../gpi_server/gpi_versions.json}

\lstinputlisting[]{../gpi_server/gpi_app.js}

% % = = = = = = = = = = = = = = = =

\section{Клиент проверяет обновления}

По нажати кнопки обновить программу, будет произведена проверка версий.
При различии версий будет скачан новый dll и обновлен файл с версиями (gpi\_versions.json). 

\lstinputlisting[language=c++]{../gpi_osisp5_lab4/gpi_osisp5_option5/gpi_updater.cpp}

\lstinputlisting[language=c++]{../gpi_osisp5_lab4/gpi_osisp5_option5/gpi_downloader.hpp}

\lstinputlisting[language=c++]{../gpi_osisp5_lab4/gpi_osisp5_option5/gpi_downloader.cpp}

\section{Новый уровень}

Новый уровень записан в файле helper.dll.

\lstinputlisting[language=c++]{../gpi_osisp5_lab4/gpi_helper/gpi_helper.h}

\lstinputlisting[language=c++]{../gpi_osisp5_lab4/gpi_helper/gpi_helper.cpp}

\paragraph{} \hspace{0pt}

\textbf{Вывод}: Ознакомился с возможностями Qt для межсерверного взаимодействия.
Научился работать с JSON файлами.
Научился отправлять запросы GET на сервер.
Научился скачивать файл с сервера.
