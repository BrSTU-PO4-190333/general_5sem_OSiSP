Цель работы:
приобрести практические навыки проектирования и разработки приложений с графическим
пользовательским интерфейсом в ОС Windows средствами Qt.

\paragraph{Вариант 5}\hspace{0pt}

Игра <<Сокобан>>.
Один уровень игры.
Общая идея: имеется комната-лабиринт (15Х15 ячеек),
в которой необходимо расставить ящики (5 штук) на указанные позиции.
Главный герой может лишь толкать ящики вперед.
Таким образом, возможны конфигурации, из которых не возможно построить желаемое решение
(например, если ящик был задвинут в тупик).

\begin{figure}[!hp]
    \centering
    \includegraphics[width=5cm]
    {../gpi_osisp5_option5/_assets/gpi_MainWindow__favicon.png}
    \caption{Иконка игры}
\end{figure}

\begin{figure}[!hp]
    \begin{minipage}{0.24\textwidth}
        \centering
        \includegraphics[width=3cm]
        {../gpi_osisp5_option5/_assets/gpi_GameWindow__floor.png}
        \caption{Пустая ячейка}
    \end{minipage}
    \begin{minipage}{0.24\textwidth}
        \centering
        \includegraphics[width=3cm]
        {../gpi_osisp5_option5/_assets/gpi_GameWindow__box.png}
        \caption{Ячейка с коробкой}
    \end{minipage}
    \begin{minipage}{0.24\textwidth}
        \centering
        \includegraphics[width=3cm]
        {../gpi_osisp5_option5/_assets/gpi_GameWindow__player.png}
        \caption{Ячейка с игроком}
    \end{minipage}
    \begin{minipage}{0.24\textwidth}
        \centering
        \includegraphics[width=3cm]
        {../gpi_osisp5_option5/_assets/gpi_GameWindow__wall.png}
        \caption{Ячейка со стеной}
    \end{minipage}
\end{figure}

\begin{figure}[h]
    \begin{minipage}{0.24\textwidth}
        \centering
        \includegraphics[width=3cm]
        {../gpi_osisp5_option5/_assets/gpi_GameWindow__finish.png}
        \caption{Финиш для коробки}
    \end{minipage}
    \begin{minipage}{0.24\textwidth}
        \centering
        \includegraphics[width=3cm]
        {../gpi_osisp5_option5/_assets/gpi_GameWindow__finBox.png}
        \caption{Коробка на финише}
    \end{minipage}
    \begin{minipage}{0.24\textwidth}
        \centering
        \includegraphics[width=3cm]
        {../gpi_osisp5_option5/_assets/gpi_GameWindow__finPlayer.png}
        \caption{Игрок на финише}
    \end{minipage}
    \begin{minipage}{0.24\textwidth}
        \centering
        \includegraphics[width=3cm]
        {../gpi_osisp5_option5/_assets/gpi_GameWindow__err.png}
        \caption{Ячейка без функционала}
    \end{minipage}
\end{figure}

\newpage

\lstinputlisting[language=make]{../gpi_osisp5_option5/gpi_osisp5_option5.pro}
\lstinputlisting[language=c++]{../gpi_osisp5_option5/main.cpp}
\lstinputlisting[language=c++]{../gpi_osisp5_option5/gpi_mainwindow.hpp}
\lstinputlisting[language=c++]{../gpi_osisp5_option5/gpi_mainwindow.cpp}
\lstinputlisting[language=c++]{../gpi_osisp5_option5/gpi_gamewindow.hpp}
\lstinputlisting[language=c++]{../gpi_osisp5_option5/gpi_gamewindow.cpp}
\lstinputlisting[]{../gpi_osisp5_option5/gpi_gamewindow.qrc}
\lstinputlisting[language=c++]{../gpi_osisp5_option5/gpi_gamewindow__gpi_drawAxes.cpp}
\lstinputlisting[language=c++]{../gpi_osisp5_option5/gpi_gamewindow__gpi_drawTextures.cpp}
\lstinputlisting[language=c++]{../gpi_osisp5_option5/gpi_gamewindow__gpi_generateLevel1.cpp}
\lstinputlisting[language=c++]{../gpi_osisp5_option5/gpi_gamewindow__gpi_goBottom.cpp}
\lstinputlisting[language=c++]{../gpi_osisp5_option5/gpi_gamewindow__gpi_goLeft.cpp}
\lstinputlisting[language=c++]{../gpi_osisp5_option5/gpi_gamewindow__gpi_goRight.cpp}
\lstinputlisting[language=c++]{../gpi_osisp5_option5/gpi_gamewindow__gpi_goTop.cpp}
\lstinputlisting[language=c++]{../gpi_osisp5_option5/gpi_gamewindow__gpi_sayWon.cpp}
